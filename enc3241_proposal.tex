\documentclass[a4paper,10pt]{article}

\usepackage{multicol}

%set margin manually since multicol{2} has strange margins
\usepackage[margin=1in]{geometry}

%use sans font
\renewcommand*{\familydefault}{\sfdefault}


\begin{document}
\title{Proposal for Fly Ash Disposal}
\date{December 2, 2014}
\author{To: Vern Fenkel\\Written by Zubir Husein on behalf of Sol Weidman and Farnsworth Inc.}
\maketitle

%need smaller than 10pt font
\small

\begin{multicols*}{2}

\subsection*{Summary}

Farnsworth is requesting permits to begin pelletization of its suitable fly ash production so it can be sold as
fertilizer and dump the remaining unsuitable fly ash in our Type III landfill. We believe our fly ash poses no risk to
the environment and that our solution is the most reasonable choice.

\subsection*{Introduction}
Farnsworth owns a forty acre Type III landfill. Historically this lot has been used by local residents to dispose of
general trash and waste and by Farnsworth to dump our waste fly ash for years. After complaints from residents,
Farnsworth and any citizens who dumped waste there received citations from the DNR. We seized our dumping and began to
store the fly ash on our property behind Shed 3A.
We strongly believe in safe and environmentally friendly manufacturing, and our company’s track record demonstrates
this. One of our concerns when we were disposing our fly ash in this lot were leachates draining into Lake Tituba.
Two months ago we hired the consulting firm Terra Engineering to analyse the land surrounding our landfill. Their
results showed that there were no detectable levels of alkalis from the fly ash in the groundwater beneath the landfill
area. In addition, they found that the groundwater flows away from the lake, so we've come to the conclusion that there
is no risk of pollution to Lake Tituba. We've also contacted fourteen other wood product companies to inquire about
their fly ash disposal practices. From these, Ashland Products has approval from the DNR to dispose of their fly ash in
a Type III landfill. Freedom Power Co. has a permit to dispose of their compacted fly ash on their unprepared property.
Richland Co. dumps their coal fly ash in a Type III landfill.
Thus we are requesting a permit to convert half of our fly ash production into pellet fertilizer and to dispose the rest
of our future fly ash in our Type III landfill in accordance to the rest of our proposed solution.

\subsection*{Proposed Solution}
We propose to pelletize three to five hundred tons of fly ash per year to be sold as low-grade garden fertilizer, and we
are requesting a permit from the DNR to do so. The remaining ash, which is about five to seven hundred tons, must be
disposed of since it is not suitable for fertilizer use. We are requesting a permit from the DNR to dump the remaining
ash on our Type III landfill. 
This will allow us to turn what otherwise would go into a landfill into a useful gardening product. It also prevents us
from having to haul over a thousand tons of fly ash 35 miles to the nearest Type II landfill and instead allows us to
dispose of it right next to us on our own property. This is much more efficient and saves fuel, money, and effort. As
indicated by the geological report, there is no risk of pollution to Lake Tituba and no indication of pollution to the
groundwater. Pittman Products, Inc. have told us that they’ve contacted the DNR about selling their fly ash as
fertilizer and they stated there were no issues with that approach. Thus we believe that our solution is very reasonable
as it poses no risk to the environment and saves the cost and effort of hauling thousands of tons to the nearest Type II landfill.

\subsection*{Method}
After obtaining approval, we will:
\begin{enumerate}
	\item Obtain a sieve to separate fertilizer suitable fly ash from the fertilizer unsuitable fly ash, which is
	 	  unsuitable due to the fused sand that comes off the furnace walls.
	\item Partition the fly ash currently on our property into fertilizer suitable and fertilizer unsuitable sections.
	\item Move the fertilizer unsuitable ash to the landfill.
	\item Obtain an extruder and clay supply in order to begin pelletization of fly ash into fertilizer.
	\item Train/hire employees to operate extruder equipment.
	\item Set up a packaging and distribution system with surrounding gardening stores in order to market the
	fertilizer.
\end{enumerate}

\subsection*{Schedule}
\bgroup
%make table less cramped
\def\arraystretch{1.5}

\begin{tabular}{| l | p{6cm} |}
\hline
Week 1 & Put in an order for sieve equipment, and an extruder. Contact local gardening stores about our fertilizer. 
Look for employees to operate the extruder.\\
\hline
Week 2 & Begin separation of the fly ash currently on our property into suitable and unsuitable portions. 
Move the unsuitable fly ash to the landfill. \\ 
\hline
Week 3 & Install extruder. Install sieve to automatically separate ash into the extruder hopper. Train personnel.\\ \hline
Week 4 & Begin fertilizer production. \\
\hline
\end{tabular}

\egroup

\end{multicols*}
\end{document}
