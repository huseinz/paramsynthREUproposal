\documentclass[a4paper,10pt]{article}

\usepackage[width=14cm, height=25cm]{geometry}
\begin{document}
\title{\vspace{-1.0cm}REU Parameter Synthesis Proposal}
\date{\small March 25, 2015}
\author{\small Written by Zubir Husein with the guidance of Faraz Hussain and Dr. Sumit K. Jha
	\\\em \small EECS Department, University of Central Florida}
\maketitle

\section*{Introduction}

When creating computational models designed to simulate complex natural phenomena that contain elements of randomness, it is often the case that a stochastic model is used to describe the system of interest. 
These models are referred to as parameterized probabilistic complex computational $(P^2C^2)$ models.
Given a set of parameters, they produce a probability distribution of possible behaviors
rather than a single deterministic output. Typically we desire to discover which set of parameters will produce a given behavior within a given probability. Traditionally this was done via brute force evaluation of all possible parameters. 
For models with many parameters it becomes computationally expensive, impractical, or impossible to use such a technique.
Thus the development of software algorithms that can intelligently discover these parameters and use of modern parallel processing techniques are of utmost importance to researchers. 

\section*{Background}

* background on the biological model we're using and the algorithms themselves *


\section*{Method}
\begin{enumerate}
	\item Create a tool implementing the algorithm in C++ that utilizes Open MPI\footnote{http://www.open-mpi.org/} and the GNU Scientific Library\footnote{http://gnu.org/software/gsl/}.
	\item Verify the correctness of the tool, do benchmarking.
	\item Create an easy to use interface for the biologists who will ultimately use the tool in their research.
	\item ...
\end{enumerate}

\section*{Schedule}

todo

\begin{thebibliography}{9}

\bibitem{hussain_1}
Faraz Hussain, Qi Mi, Joyeeta Dutta-Moscato, Yoram Vodovotz and Sumit K. Kha.
A new algorithmic technique for learning parameters in computational systems biology: applications to an acute
inflammatory response model. 2014
\bibitem{jha_1}
Sumit K. Jha. Exascale algorithms for synthesizing parameters of stochastic computational models from qualitative and semi-quantitative specifications. 2012


\end{thebibliography}



\end{document}
